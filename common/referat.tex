%\newcommand{\actuality}{\underline{\textbf{Актуальность темы.}}}
%\newcommand{\development}{\underline{\textbf{Степень разработанности.}}}
%\newcommand{\aim}{\underline{\textbf{Целью}}}
%\newcommand{\tasks}{\underline{\textbf{задачи}}}
%\newcommand{\methods}{\underline{\textbf{Методы исследования. }}}
%\newcommand{\novelty}{\underline{\textbf{Научная новизна:}}}
%\newcommand{\theorInfluence}{\underline{\textbf{Теоретическая значимость.}}}
%\newcommand{\influence}{\underline{\textbf{Практическая значимость.}}}
%\newcommand{\defpositions}{\underline{\textbf{Основные положения, выносимые на~защиту:}}}
%\newcommand{\reliability}{\underline{\textbf{Достоверность}}}
%\newcommand{\probation}{\underline{\textbf{Апробация работы.}}}
%\newcommand{\contribution}{\underline{\textbf{Личный вклад.}}}
%\newcommand{\publications}{\underline{\textbf{Публикации.}}}

\textbf{Актуальность исследования.}
Устройства записи, обработки и воспроизведения цифровых изображений и видео высокого качества становятся все более популярны у широкого круга пользователей. Одним из них является смартфон, с которым связаны разные стороны деятельности человека, что приводит к так называемому взрыву данных: изображений и видео. Это инициировало и стимулирует развитие интеллектуального анализа данных, разработку приложений для анализа и обработки изображений и видео. Возникают новые задачи, которые требует развития областей компьютерного зрения для проектирования и разработки все более сложных и более высокопроизводительных приложений. Такие задачи требуют разработки лучших методов, чтобы стать ближе к реальным потребностям людей.

Компьютерное зрение (КЗ) включает в себя методы регистрации, обработки, анализа и классификации изображений. КЗ описывается как процесс автоматизации, интеграции и распознавания визуальной информации (В.С. Сизиков - 2001, 2012), (Р. Гонсалес - 2006). КЗ плотно связано с теорией искусственного интеллекта с извлечением информации из изображений. Данные изображений могут быть получены из многих источников, таких как: видеоизображения с нескольких камер, многомерные данные с медицинских томографов и др. Применение моделей и теории для построения систем компьютерного зрения является важной областью развития компьютерных наук.

Автоматизация антропометрических измерений – важная область приложения компьютерного зрения. Здесь можно применять компьютерное зрение для получения и анализа данных о форме, размерах человеческого тела. Поэтому исследования в области антропометрических признаков хорошо развиваются. Размеры человеческого тела являются важными объектами исследования во многих научных исследованиях. Чтобы понять содержание изображения, нужно извлечь и классифицировать признаки объекта. Эти задачи представляет большой интерес для исследователей.

Для автоматизации антропометрии необходимо определить ключевые точки. Это один из основных этапов процесса анализа. Задачей настоящего исследования является совершенствование и разработка алгоритмов для обработки изображений и видео для обеспечения точности при условии наличия шума и работающей в режиме реального времени. В силу ограниченности ресурсов смартфонов является важным, чтобы система компьютерного зрения стабильно работала с высокой эффективностью.

Добиться этого можно используя методы компьютерного зрения. Для создания эффективных приложений необходимо использовать и расширенные возможности интеллектуальной системы компьютерного зрения.

Отметим, что антропометрические признаки можно применять в области здравоохранения, пошива одежды и фитнеса. Кроме того, в области развлечений, безопасности и мобильных приложений также есть интерес к получению антропометрической информации.

\textbf{Целью исследования} является применение численных методов компьютерного зрения и математического моделирования в антропометрии, реализация комплекса программ в виде приложения для смартфонов и апробация на практике. Для достижения указанной цели были поставлены следующие \textbf {основные задачи}:
\begin{itemize}
	\item Разработка алгоритмов и методов компьютерного зрения для извлечения антропометрических признаков из изображений и видео в режиме реального времени при наличии шума;
	\item Объединение алгоритмов и методов компьютерного зрения для достижения высокой эффективности и повышения точности извлечения антропометрических признаков;
	\item Применение методов машинного обучения для классификации данных антропометрических признаков;
	\item Разработка метода построения 3D модели человеческого тела. Этот метод требует правильное описание структуры и формы человека;
	\item Разработка антропометических приложений для смартфонов с операционной системой Андроид для использования в текстильной промышленности и в фитнесе;
	\item Оценка качества и эффективности работы системы компьютерного зрения в антропометрии на видео в среде Андроид.
\end{itemize}
Результаты проведенных экспериментов подтвердили эффективность алгоритмов компьютерного зрения в антропометрии и сравнение с другими соответствующими исследованиями подтвердило точность и устойчивость предложенного подхода и комплекса программ. 

\textbf{Внедрение работы.} Результаты исследования были применены на практике в области текстильной промышленности в компании «ШиК, ателье по пошиву военной формы» города Иркутска. Программа была представлена экспертам в этой области и были получены сертификаты, подтверждающие точность измерений и работоспособность приложения на практике.

\textbf{Предмет исследования.} Предмет исследования определен предметной областью №7 паспорта специальности 05.13.18, «Разработка новых математических методов и алгоритмов интерпретации натурного эксперимента на основе его математической модели», а так же перечнем задач решаемых в диссертации.

\textbf{Методы исследования.} Методы  теоретических исследований: Алгоритмы и методы компьютерного зрения в антропометрии; Методы анализа данных и построения 3D-моделей. Методы прикладных исследований: Проектирование алгоритмов для задачи извлечения признаков и классификации антропометрических признаков. Разработка 3D моделей для моделирования формы человеческого тела; Построение приложения на операционной системе Андроид; Тестирование программы и хранение результатов, оценка и сравнение результатов с другими методами и алгоритмами.

\textbf{Теоретической значимостью} результатов исследования диссертационной работы является разработка и тестирование новых моделей в антропометрии на основе сочетания алгоритмов и методов компьютерного зрения в антропометрии на основе обработки видеопотока.

\textbf{Научная ценность исследования}. Исследование позволило расширить область приложения методом компьютерного зрения в задачах антропометрии. Исследования показали, что сочетание алгоритма сегментации изображений на основе разреза графов с  итеративным алгоритмом ближайших точек -  ICP повысило точность процесса извлечения антропометрических признаков. Кроме того, исследования показали, что алгоритм случайного леса позволяет эффективно решать задачу классификации антропометрических признаков.

\textbf{Научная новизна} результатов диссертационной работы заключается в следующем:

\begin{enumerate}
	\item[1)] Предложены методы, алгоритмы компьютерного зрения для извлечения антропометрических признаков , основанные на сочетании двух алгоритмов сегментации изображения: разреза графов и алгоритма итеративных ближайших точек (Iterative Closest Point — ICP). Здесь цель состояла в том, чтобы получить полный набор ключевых точек, описывающих размеры человеческого тела;
	\item[2)] Приложение методов машинного обучения на основе случайного леса для классификации антропометрических измерений метода;
	\item[3)] Разработка методов построения антропометрических моделей человеческого тела на основе антропометрических признаков полученных при помощи авторских методов компьютерного зрения;
	\item[4)] Разработка бесконтактной системы антропометрии для смартфона на операционной системе Андроид.
\end{enumerate}

\textbf{Практическая ценность} заключается в решении задачи антропометрии на базе смартфонов. Результаты внедрены для автоматизации процедуры профессионального снятия мерок для пошива одежды, а также для фитнес-тестирования.

\textbf{Апробация работы.} Работа выполнялась на кафедре вычислительной техники ИРНИТУ. Результаты диссертационной работы обсуждались и докладывались на следующих симпозиумах, семинарах и конференциях: Всероссийские молодежные научно-практические конференции «Винеровские чтения» (Иркутск, 2014, 2015, ИРНИТУ); XIX Байкальская всероссийская конференция «Информационные и математические технологии в науке и управлении» (Улан-Удэ, 2014); The 4th, 5th International Conference on Analysis of Images, Social Networks, and Texts (Екатеринбург, 2015, 2016); V Научно-практическая Internet-конференция «Междисциплинарные исследования в области математического моделирования и информатики» (Тольятти, 2015).

Результаты диссертации неоднократно докладывались на научных семинарах кафедры вычислительной техники Иркутского национального исследовательского технического университета и Института систем энергетики им. Л.А. Мелентьева СО РАН.

\textbf{Личный вклад автора.} Основные результаты выносимые на защиту получены автором лично. Конфликта интересов с соавторами нет.

\textbf{Публикации.} По теме диссертации опубликовано 11 научных работ, 3 из которых – в рецензируемых научных журналах и изданиях, рекомендованных ВАК РФ, 2 свидетельства регистрации программы на ЭВМ, одна статья опубликована в журнале, включенном в перечень Web of Science.

\textbf{Структура и объем работы.} Диссертация содержит введение, четыре главы, заключение и список использованной литературы, содержащий 151 наименований. Общий объем диссертации составляет 123 страниц машинописного текста, иллюстрированного 54 рисунками и 5 таблицами.
