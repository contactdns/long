%\newcommand{\actuality}{\underline{\textbf{Актуальность темы.}}}
%\newcommand{\development}{\underline{\textbf{Степень разработанности.}}}
%\newcommand{\aim}{\underline{\textbf{Целью}}}
%\newcommand{\tasks}{\underline{\textbf{задачи}}}
%\newcommand{\methods}{\underline{\textbf{Методы исследования. }}}
%\newcommand{\novelty}{\underline{\textbf{Научная новизна:}}}
%\newcommand{\theorInfluence}{\underline{\textbf{Теоретическая значимость.}}}
%\newcommand{\influence}{\underline{\textbf{Практическая значимость.}}}
%\newcommand{\defpositions}{\underline{\textbf{Основные положения, выносимые на~защиту:}}}
%\newcommand{\reliability}{\underline{\textbf{Достоверность}}}
%\newcommand{\probation}{\underline{\textbf{Апробация работы.}}}
%\newcommand{\contribution}{\underline{\textbf{Личный вклад.}}}
%\newcommand{\publications}{\underline{\textbf{Публикации.}}}

\textbf{Актуальность исследования.}
Автоматизация антропометрических измерений \pdfcomment[avatar=EA,type=line]{Может быть дать определение как измерение, а не часть области КЗ} -- важная область приложения компьютерного зрения. Здесь можно применять компьютерное зрение для получения и анализа данных о форме, размерах человеческого тела. Поэтому исследования в области антропометрических признаков хорошо развиваются. Размеры человеческого тела являются важными объектами исследования во многих научных исследованиях \cite{Paul2011,Michaell2011,Wuhre2014}. Чтобы понять содержание изображения, нужно извлечь и классифицировать признаки объекта. Эти задачи представляет большой интерес для исследователей.

Извлечению антропометрических признаков посвящено много успешных исследований. Например, для статических изображений были выполнены научно-исследовательские работы обнаружения человеческого тела, используя описание модели человеческого тела \cite{YuChen2011, Albio2001}. Barron и Kakadiaris \cite{Barron2000} и Taylors \cite{Taylor2000} изучали восстановление 3D – модели человеческого тела. Ещё одна группа решений для антропометрии представлена в работах \cite{Micilotta2005, Mikolajczyk2004, Robert2004, Ronfard2002}. Однако эти методы позволяют находить только основные части тела человека. Более того, задача автоматизации антропометрии ими не рассматривалась.

Результаты таких исследований позволяют разрабатывать эффективные методы мониторинга движения (Liu 2010 \cite{Liu2010}, Wang  2013 \cite{Wang2013},  Yan 2014 \cite{Yan2014}). Имеется цикл работ по распознаванию действия людей и их обнаружение (см. работы Pirsivash и Ramanan  2012 \cite{Pirsiavash2012}, Gan 2015 \cite{Gan2015}, Jainy 2015 \cite{Jainy2015}, Jordi \cite{Jordi2012}, Konstantinos \cite{Konstantinos2016}), где система способна локализовать деятельность человека на изображении.

Добиться этого можно используя методы компьютерного зрения. Для создания эффективных приложений необходимо использовать и расширенные возможности интеллектуальной системы компьютерного зрения.

Компьютерное зрение (КЗ) включает в себя методы регистрации, цифровой обработки, анализа и идентификации изображений. КЗ описывается как процесс автоматизации, интеграции и распознавания визуальной информации \cite{Wang2016,Maria2016,Ioan2011,Lauren2013,Saha2005,Azriel2001}. КЗ связано с теорией искусственного интеллекта с извлечением информации из изображений. Данные изображений могут быть получены из многих источников, таких как: видеоизображения с нескольких камер, многомерные данные с медицинских томографов и др. Применение моделей и теории для построения систем компьютерного зрения является важной областью развития компьютерных наук.

В современной жизни устройства записи, обработки и воспроизведения цифровых изображений и видео высокого качества стали популярны для широкого круга пользователей. Одним из них является смартфон, с которым связаны разные стороны деятельности человека, что приводит к так называемому взрыву данных: изображений и видео. Это инициировало и стимулирует развитие интеллектуального анализа данных, разработку приложений анализа и обработки изображений и видео. В этих областях возникают новые задачи, которые требует развития областей компьютерного зрения для проектирования и разработки все более интеллектуальных и более высокопроизводительных приложений. Это задача привлечения научных инноваций, лучших методов, чтобы стать ближе к реальным потребностям в различных областях.

Для автоматизации антропометрии необходимо определить ключевые точки. Это один из основных этапов процесса анализа. Задачей настоящего исследования является совершенствование и разработка алгоритмов для обработки изображений и видео для обеспечения точности при условии наличия шума и работающей в режиме реального времени. В силу ограниченности ресурсов смартфонов является важным, чтобы система компьютерного зрения стабильно работала с высокой эффективностью.

Отметим, что антропометрические признаки можно применять в области здравоохранения, пошива одежды и фитнеса. Кроме того, в области развлечений, безопасности и мобильных приложений также есть интерес к получению антропометрической информации.

\textbf{Целью исследования} является применение численных методов компьютерного зрения и математического моделирования в антропометрии, реализация комплекса программ в виде приложения для смартфонов. Для достижения указанной цели были поставлены следующие \textbf {основные задачи}:
\begin{itemize}
	\item Разработка алгоритмов и методов компьютерного зрения для извлечения антропометрических признаков из изображений и видео в режиме реального времени при наличии шума;
	\item Объединение алгоритмов и методов компьютерного зрения для достижения высокой эффективности и повышения точности извлечения антропометрических признаков;
	\item Применение методов машинного обучения для классификации данных антропометрических признаков;
	\item Разработка способа построения антропометрических 3D-моделей человеческого тела. Этот метод требует правильное описание структуры и формы человека;
	\item Разработка антропометических приложений для смартфонов с операционной системой Андроид для использования в текстильной промышленности и в фитнесе;
	\item Оценка качества и эффективности работы системы компьютерного зрения в антропометрии на видео в среде Андроид.
\end{itemize}
Результаты проведенных экспериментов подтвердили эффективность алгоритмов компьютерного зрения в антропометрии и сравнение с другими соответствующими исследованиями подтвердило точность и устойчивость предложенного подхода и комплекса программ.

\textbf{Внедрение работы.} Результаты исследования были применены на практике в области текстильной промышленности в компании «ШиК, ателье по пошиву военной формы» города Иркутска. Программа была представлена экспертам в этой области и были получены сертификаты, подтверждающие точность измерений и работоспособность приложения на практике.

\textbf{Предмет исследования.} Предмет исследования определен предметной областью №7 паспорта специальности 05.13.18, «Разработка новых математических методов и алгоритмов интерпретации натурного эксперимента на основе его математической модели», а так же перечнем задач решаемых в диссертации.

\textbf{Методы исследования.}
\begin{itemize}
	\item Методы  теоретических исследований:

	\begin{itemize}
		\item Алгоритмы и методы компьютерного зрения в антропометрии;
		\item Методы анализа данных и построения 3D-моделей.
	\end{itemize}

\end{itemize}
\begin{itemize}
	\item Методы прикладных исследований:

	\begin{itemize}
		\item Проектирование алгоритмов для задачи извлечения признаков и классификации антропометрических признаков. Разработка 3D моделей для моделирования формы человеческого тела;
		\item Построение приложения на операционной системе Андроид;
		\item Тестирование программы и хранение результатов, оценка и сравнение результатов с другими методами и алгоритмами.
	\end{itemize}

\end{itemize}

\textbf{Теоретической значимостью} результатов исследования диссертационной работы является разработка и тестирование новых моделей в антропометрии на основе сочетания алгоритмов и методов компьютерного зрения в антропометрии на основе обработки видеопотока.

\textbf{Научная ценность исследования}. Исследование позволило расширить область приложения методом компьютерного зрения в задачах антропометрии. Исследования показали, что сочетание алгоритма сегментации изображений на основе разреза графов с  итеративным алгоритмом ближайших точек -  ICP повысило точность процесса извлечения антропометрических признаков. Кроме того, исследования показали, что алгоритм случайного леса позволяет эффективно решать задачу классификации антропометрических признаков.

\textbf{Научная новизна} результатов диссертационной работы заключается в следующем:

\begin{enumerate}
	\item[1)] Предложены методы, алгоритмы компьютерного зрения для извлечения антропометрических признаков , основанные на сочетании двух алгоритмов сегментации изображения: разреза графов и алгоритма итеративных ближайших точек (Iterative Closest Point — ICP). Здесь цель состояла в том, чтобы получить полный набор ключевых точек, описывающих размеры человеческого тела;
	\item[2)] Приложение методов машинного обучения на основе случайного леса для классификации антропометрических измерений метода;
	\item[3)] Разработка методов построения антропометрических моделей человеческого тела на основе антропометрических признаков полученных при помощи авторских методов компьютерного зрения;
	\item[4)] Разработка бесконтактной системы антропометрии для смартфона на операционной системе Андроид.
\end{enumerate}

\textbf{Практическая ценность} заключается в решении задачи антропометрии на базе смартфонов. Результаты внедрены для автоматизации процедуры профессионального снятия мерок для пошива одежды, а также для фитнес-тестирования.

\textbf{Апробация работы.} Работа выполнялась на кафедре вычислительной техники ИРНИТУ. Результаты диссертационной работы обсуждались и докладывались на следующих симпозиумах, семинарах и конференциях: Всероссийские молодежные научно-практические конференции «Винеровские чтения» (Иркутск, 2014, 2015, ИРНИТУ); XIX Байкальская всероссийская конференция «Информационные и математические технологии в науке и управлении» (Улан-Удэ, 2014); The 4th, 5th International Conference on Analysis of Images, Social Networks, and Texts (Екатеринбург, 2015, 2016); V Научно-практическая Internet-конференция «Междисциплинарные исследования в области математического моделирования и информатики» (Тольятти, 2015).

Результаты диссертации неоднократно докладывались на научных семинарах кафедры вычислительной техники Иркутского национального исследовательского технического университета и Института систем энергетики им. Л.А. Мелентьева СО РАН.


\textbf{Личный вклад автора.} Основные результаты выносимые на защиту получены автором лично. Конфликта интересов с соавторами нет.

\textbf{Тематика работы} соответствует следующим пунктам паспорта специальности 05.13.18:
\begin{itemize}
	\item п.3 <<Разработка, обоснование и тестирование эффективных вычислительных методов с применением современных компьютерных технологий>>.
	\item п.4. <<Реализация эффективных численных методов и алгоритмов в виде комплексов проблемно-ориентированных программ для проведения вычислительного эксперимента>>.
	\item п.5 <<Комплексные исследования научных и технических проблем с применением современной технологии математического моделирования и вычислительного эксперимента>>.
\end{itemize}

\textbf{Публикации.} По теме диссертации опубликовано 11 научных работ, 3 из которых – в рецензируемых научных журналах и изданиях, рекомендованных ВАК РФ, 2 свидетельства регистрации программы на ЭВМ, одна статья опубликована в журнале, включенном в перечень Web of Science.

Основные результаты исследования опубликованы в следующих работах.

\textbf{Издания, входящие в Перечень ВАК РФ:}

\begin{enumerate}
	\item Нгуен~Т.Л. Об автоматизации извлечения и классификации антропометрических признаков/ Нгуен Т.Л., Нгуен Т.Х. // Вестник ИРНИТУ: №~4. 2015.-С. 17-23.
	\item Нгуен~Т.Л. О распознавании и классификации дефектов дорожного покрытия на основе изображений // Нгуен Т.Л., Нгуен Т.Х.// Вестник ИРНИТУ: №~10. 2016. -С. 111-118.
	\item Nguyen~T.L. Automatic Anthropometric System Development Using Machine Learning / Nguyen T. L., Nguyen T.H.// BRAIN. Broad Research in Artificial Intelligence and Neuroscience. 2016, V. 7, P. 5-15.\\
	\textbf{Издания, включенные в РИНЦ:}
	\item Nguyen T.H. A Robust Approach for Defects Road Pavement Detection and Classification/ Nguyen T. L., Nguyen T.H., D. N. Sidorov // Journal of Computational and Engineering Mathematics: 2016,-V. 3.-No. 3. P. 40-52.\\
	\textbf{Свидетельства о государственной регистрации программы для ЭВМ:}
	\item  Сидоров Д.Н. Программа бесконтактной антропометрии для смартфонов на операционной системе Андроид // Сидоров Д.Н., Нгуен Т.Л., Нгуен Т.Х. // Свидетельство о гос. регистрации программы для ЭВМ. № 2016611475, от 03 февраля 2016 г. М.: Федеральная служба по интеллектуальной собственности. 2016.
	\item  Сидоров Д.Н. Программа бесконтактной антропометрии для смартфонов на операционной системе Андроид // Сидоров Д.Н.,  Нгуен Т.Х., Нгуен Т.Л. // Свидетельство о гос. регистрации программы для ЭВМ. № 2016619386, от 18 августа  2016 г. М.: Федеральная служба по интеллектуальной собственности. 2016.\\
\textbf{Прочие издания:}
\item  Нгуен Т.Л. Автоматизация антропометрических измерений и извлечение признаков из 2D-изображений / Нгуен Т.Л., Нгуен Т.Х. // Байкальская международная школа-семинар <<методы оптимизации и их приложения>>. О. Ольхон, Иркутск 2014г. -С. 153.
\item Нгуен Т.Л. Построение программы для обнаружения контуров человека в изображении с помощью методов математической морфологии / Нгуен Т.Л., Нгуен Т.Х. // Материалы всероссийской молодежной научно-практической конференции <<Винеровские чтения 2014>>. Иркутск: Изд-во Иркутск, 2014. -С 10.
\item Нгуен Т.Л. Классификация и кластерный анализ антропометрических признаков / Нгуен Т.Л.// Материалы всероссийской молодежной научно-практической конференции <<Винеровские чтения 2015>>. Иркутск: Изд-во Иркутск, 2015. -С.8.
\item Нгуен Т.Л. Методы математической морфологии в цифровой обработке изображений / Нгуен Т.Л., Нгуен Т.Х. // Труды XIX Байкальской Всероссийской конференции <<информационные и математические технологии в науке и управлении>>. Иркутск: ИСЭМ СО РАН, 2014. -С. 75-81.
\item  Nguyen~T.L. Studies of Anthropometrical Features using Machine Learning Approach / Nguyen T.L., Nguyen T.H., A. Zhukov // Supplementary Proceedings of the 4th International Conference on Analysis of Images, Social Networks and Texts (AIST 2015). CEUR Workshop Proceedings. 2015. -P. 96-105.
\item Нгуен Т.Л. Анализ антропометрических признаков с использованием методов машинного обучения / Нгуен Т.Л., Нгуен Т.Х. // Междисцплинарные исследования в области математического моделирования и информатики . Ульяновск: Изд-во SIMJET, января 2015г.-С.204-210.
\item Nguyen T.H. On Road Defects Detection and Classification / Nguyen T.H., Nguyen T.L., A. Zhukov // Supplementary Proceedings of the 5th International Conference on Analysis of Images, Social Networks and Texts (AIST 2016). CEUR Workshop Proceedings, 2016,-V. 1710, -P. 266- 278.
\end{enumerate}


\textbf{Структура и объем работы.} Диссертация содержит введение, четыре главы, заключение и список использованной литературы, содержащий 151 наименований. Общий объем диссертации составляет 123 страниц машинописного текста, иллюстрированного 54 рисунками и 5 таблицами.

Кратко изложим содержание \textbf{основных разделов работы}.

\textbf{В первой главе} представлен аналитический обзор алгоритмов и методов компьютерного зрения в антропометрии на основе статических изображений и видео в режиме реального времени. Приведены их преимущества и недостатки. Рассматривается возможность использования методов, алгоритмов компьютерного зрения для извлечения признаков.

\textbf{Во второй главе} излагаются принципы решения задачи антропометрии на статических изображениях и видеопоследовательностях в режиме реального времени с использованием методов компьютерного зрения. Приводится подробное описание алгоритмов и методов, используемых для обнаружения и классификации объектов, извлечения признаков из изображений и видеопоследовательностей.

\textbf{В третьей главе} представлена информация о анализе, проектировании и построении системы компьютерного зрения для решения задачи антропометрии на основе анализа объектно-ориентированного языка UML. Описание структуры библиотеки классов приложений. Анализ результатов испытаний применения алгоритмов компьютерного зрения на видео в присутствии шума и в режиме реального времени системы.

\textbf{В четвертой главе} представлены результаты разработки системы компьютерного зрения в антропометрии. Приложение разработано на операционной системе Андроид для смартфонов. Даны описание использованных библиотек, выбор инструментов для поддержки разработки прикладного программного обеспечения <<Measure Me>>. Изложено описание интерфейса.

%%% Local Variables:
%%% mode: latex
%%% TeX-master: "../Dissertation/dissertation"
%%% End:
