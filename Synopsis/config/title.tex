\thispagestyle{empty}

\vspace{0pt plus1fill} %число перед fill = кратность относительно некоторого расстояния fill, кусками которого заполнены пустые места
\begin{flushright}
  \large{На правах рукописи}
  %\includegraphics[height=1.5cm]{images/personal-signature}
\end{flushright}

\vspace{0pt plus3fill} %число перед fill = кратность относительно некоторого расстояния fill, кусками которого заполнены пустые места
\begin{center}
%\textbf
{\large \thesisAuthor}
\end{center}

\vspace{0pt plus3fill} %число перед fill = кратность относительно некоторого расстояния fill, кусками которого заполнены пустые места
\begin{center}
\textbf {\Large \thesisTitle}

\vspace{0pt plus3fill} %число перед fill = кратность относительно некоторого расстояния fill, кусками которого заполнены пустые места
{\large Специальность \thesisSpecialtyNumber\ "---\par <<\thesisSpecialtyTitle>>}

\vspace{0pt plus1.5fill} %число перед fill = кратность относительно некоторого расстояния fill, кусками которого заполнены пустые места
\textbf{\Large{Автореферат}}\par
\large{диссертации на соискание учёной степени\par \thesisDegree}
\end{center}

\vspace{0pt plus4fill} %число перед fill = кратность относительно некоторого расстояния fill, кусками которого заполнены пустые места
\begin{center}
{\large{\thesisCity\ "--- \thesisYear}}
\end{center}

\newpage
% оборотная сторона обложки
\thispagestyle{empty}
\noindent Работа выполнена в \thesisInOrganization
\vfill

\par\bigskip
%\begin{table}[h] % считается не очень правильным использовать окружение table, не задавая caption
    \noindent%
    \begin{tabular}{@{}lp{11cm}}
        \sfs \textbf{Научный руководитель:} & \sfs \supervisorRegalia \par
                                      \textbf{\supervisorFio}
        \vspace{4mm} \\
        {\sfs \textbf{Официальные оппоненты:}} &
        {\sfs \textbf{\opponentOneFio,}\par
                  \opponentOneRegalia,\par
                  \opponentOneJobPlace,\par
                  \opponentOneJobPost\par \vspace{3mm}
                  \textbf{\opponentTwoFio,}\par \vspace{1mm}
                  \opponentTwoRegalia,\par
                  \opponentTwoJobPlace,\par
                  \opponentTwoJobPost
        }
        \vspace{4mm} \\
        {\sfs \textbf{Ведущая организация:}} & {\sfs \leadingOrganizationTitle }
    \end{tabular}
%\end{table}
\par
\vfill\vfill\vfill

\noindent Защита состоится \defenseDate~на~заседании диссертационного совета \defenseCouncilNumber~на базе \defenseCouncilTitle~по адресу: \defenseCouncilAddress.

\vspace{5mm}
\noindent С диссертацией можно ознакомиться в библиотеке \synopsisLibrary.

\vspace{5mm}
\noindent{Автореферат разослан \synopsisDate.}

\vfill\vfill\vfill
%\begin{table} [h] % считается не очень правильным использовать окружение table, не задавая caption
\par\bigskip%
\noindent%
    \begin{tabular}{p{10cm}cr}
        \begin{tabular}{p{10cm}}
            \sfs Ученый секретарь  \\
            \sfs диссертационного совета  \\
            \sfs \defenseCouncilNumber, \defenseSecretaryRegalia
        \end{tabular}
    &
        \begin{tabular}{c}
            %\includegraphics[height=1cm]{images/secretary-signature}
        \end{tabular}
    &
        \begin{tabular}{r}
            \\
            \\
            \sfs \defenseSecretaryFio
        \end{tabular}
    \end{tabular}
%\end{table}
\newpage

%%% Local Variables:
%%% mode: latex
%%% TeX-master: "../synopsis"
%%% End:
