\section{Основные результаты и выводы по главе 3}

\begin{enumerate}
	\item В этой главе описывается анализ проектирования систем компьютерного зрения в антропометрии для практических применений: пошив одежды и фитнес-тестирование с помощью аналитических методов объектно-ориентированного UML. Используются диаграммы прецедентов, диаграммы классов и диаграммы последовательности. Классы подробно анализируются, с указанием задач каждой компоненты в программе. Также обоснована целесообразность проектирования приложений компьютерного зрения;
	\item Описаны результаты тестирования алгоритмов компьютерного зрения, в том числе предварительной обработки изображений с помощью фильтра Гаусса, эквализация гистограммы, алгоритм вычитания фона изображения на основе фоновых модели, алгоритм сегментации изображений методом разреза на графах (graph cuts), итеративный алгоритм ближайших точек (Iterative Closest Point - ICP). Алгоритмы функционируют в режиме реального времени и в условиях наличия с шумов в видеопоследовательностях. Эксперименты доказали, что предложенные алгоритмы эффективно работают и включают в себя: извлечение и классификация антропометрических признаков;
	\item Проведен анализ практических результатов экспериментов извлечения антропометрических признаков. Проведено сравнение результатов предложенных методов с методом выпуклой оболочки и вычитания фона по точности. Доказывается преимущество приведенного алгоритма на основе метода разреза графа и итеративного алгоритма ближайших точек;
	\item Результаты экспериментов по классификации антропометрических данных на компьютере и смартфоне показали, что алгоритм случайного леса эффективно работает для решения задач классификации в видеопоследовательностях в режиме реального времени;
	\item Сравнение результатов классификации между алгоритмом случайного леса и алгоритмом Boosting, работающими с видео, показало, что по точности и времени выполнения, более результативный алгоритм случайного леса для системы компьютерного зрения в антропометрии. Алгоритм случайного леса полностью совместим с визуальной системой.
\end{enumerate}
