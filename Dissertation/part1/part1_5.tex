\section{Основные результаты и выводы по главе 2}

\begin{enumerate}
	\item Предложен метод извлечения антропометрических признаков на изображениях и видео на основе алгоритмов компьютерного зрения. Проведено улучшение качества изображения с помощью технической обработки изображений, обнаружение объектов с помощью алгоритма вычитания фона, сегментации изображения с помощью алгоритма разреза на графах, извлечения ключевых точек методом $ICP$;
 \item Предложен метод поправочного множителя с использованием комбинированного алгоритма $ICP$ для повышения точности извлечения антропометрических признаков на изображениях и видео.
\item Предложен метод классификации антропометрических данных с помощью алгоритма случайных лесов - Random Forest для классификации размеров и $3D$ моделей человеческого тела.
\item Предложена модель классификации Wrapper и алгоритм извлечения важных признаков для повышения точности и скорости алгоритма случайного леса.
\item Разработан алгоритм и схема его применения для извлечения и классификации антропометрических данных с помощью алгоритмов компьютерного зрения.
\item На основе численных экспериментов проведен сравнительный анализ применимости и сравнение возможности применения алгоритмов компьютерного зрения в антропометрии - обнаружение объектов с помощью алгоритма вычитания фона, сегментации изображения с помощью алгоритма разреза графов, извлечения ключевых точек методом $ICP$, классификации антропометрических данных с помощью алгоритма случайных лесов для решения проблемы в  антропометрии на статических изображениях в присутствии шума. 
\item Экспериментальные результаты представлены в таблицах (\ref{tab1}), (\ref{tab3}), (\ref{tab5}), (\ref{tab6}) показывают, что алгоритмы работают с высокой скоростью и точностью при извлечении и классификации антропометрических данных на изображениях и на видеопоследовательностях в присутствии шума и в реальном режиме времени. 

\end{enumerate}
