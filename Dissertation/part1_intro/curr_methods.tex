%\subsection{Существующие методы выравнивания графика нагрузки}
%\subsection{Обзор результатов исследований технологий сохранения электроэнергии}
%\subsection{Обзор результатов исследований технологий аккумулирования электроэнергии}
\section{Обзор результатов исследований технологий аккумулирования электроэнергии}

В настоящее время энергетика находится на пороге новых изменений вследствие широкого использования различных возобновляемых источников энергии в условиях свободного рынка электроэнергии.
Высокая доля различных возобновляемых источников энергии усиливает изменчивость и непостоянность выработки электроэнергии, нарушая оптимальный режим работы традиционных %надежность 
энергосистем.
Здесь может быть интересен зарубежный опыт.
Например, в странах Европейского союза, как показывает недавний прогноз развития мировой энергетики \cite{IEA1%International Energy Agency (IEA). World energy outlook 2013. Paris: OECD/IEA; 2013
}, опубликованный международным энергетическим агентством, прогнозируется значительный рост доли различных возобновляемых источников энергии в общем объеме выработки электроэнергии с 6,9\% в 2011 году до 23,1\% к 2035 году, а европейская комиссия признает аккумулирование электроэнергии в качестве одной из стратегических технологий в плане достижения целевых энергетических показателей к 2020 и 2050 гг. \cite{EuroCom%European Commission. Strategic energy technologies [online]. Available: http://setis.ec.europa.eu/technologies;2013
}.
Министерство энергетики США также определило аккумулирование энергии в качестве решения задачи обеспечения стабильности сети через программу аккумулирования энергии для создания систем и технологий аккумулирования электроэнергии \cite{Sandia%Sandia National Laboratories. Energy storage systems program [online]. Available:http://www.sandia.gov/ess/;2013
}.
Стоит отметить, что широкий спектр исследований направлен на изучение технических характеристик технологий накопления электрической энергии.
Например, технические характеристики различных систем ЭЭС были предметом исследования и рассмотрения в ряде работ \cite{Evans20124141,IEA5,Hall20084352,Hadjipaschalis20091513,Ibrahim20081221,Chen2009291,Gazarian2011,Baker20084368,DiazGonzalez20122154}.

Есть и другие исследования, которые были более тщательно исследованы эксплуатационные характеристики некоторых технологий аккумулирования электроэнергии, включая гидроаккумулирующие электростанции \cite{Punys2013190}, аккумулирование энергии с помощью сжатого воздуха \cite{Karellas2014865}, различные типы аккумуляторных батарей \cite{Dunn2011,Poullikkas2013778,Alotto2014325}, аккумулирование энергии с помощью маховиков \cite{Sebastian20126803,Bolund2007235}, сверхпроводящие магнитные накопители энергии \cite{Ali2010}, накопителей энергии в виде суперконденсаторов \cite{Noriega2013}.
Также большое количество исследований проводится по моделированию и оптимизации хранения электрической энергии на примере экспериментальных или реальных энергосистем \cite{Steffen2013556,Rugolo2012,DiSilvestre2013,Makarov2012,Barbour2012,Evans2013405}. %,27_,29_,30_}.
Данные исследования способствуют лучшему пониманию текущего состояния исследований ЭЭС с использованием накопителей электроэнергии, их технические характеристики, функциональные ограничения и возможности проектирования систем хранения электроэнергии.
Достаточно большое внимание уделяется анализу производительности, технических характеристик и стоимости использования батарей.
Отметим, что большинство работ предполагают использование констант в качестве определения %коэффициентов  %эффективности 
%полезного действия 
КПД
используемых накопителей энергии.

Также необходимо заметить, что в настоящий момент не существует комплексной методики определения графика нагрузки дополнительных генерирующих мощностей с использованием аккумулирования электроэнергии, %с учетом 
учитывающей различные %коэффициенты %эффективности 
%полезного действия %энергосберегающих технологий
КПД
накопителей энергии, которые могут изменяться со временем в силу различного рода причин: специфики технологии использования, факторов внешней среды, изношенности, временной замены более эффективных модулей менее эффективным (и наоборот, например, во время ремонта) и т.д.
Таким образом, для устранения ограничений в развертывании крупномасштабных систем хранения электроэнергии в существующих электроэнергетических системах необходимы дальнейшие исследования.