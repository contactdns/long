\section{Цель и задачи исследования}

Система компьютерного зрения в антропометрии для смартфонов включает следующие три основных этапа:
\begin{itemize}
	\item Первый этап: извлечение антропометрических признаков и создание векторов антропометрических признаков;
	\item Второй этап: классификация данных антропометрических признаков на основе данных из векторов признаков;
	\item Третий этап: Создание приложений для смартфонов на базе результатов классификации.
\end{itemize}

В данной работе осуществляется разработка алгоритмов, методов компьютерного зрения для извлечения признаков и классификации данных в статических изображениях и видео в режиме реального времени. Она актуальна из-за необходимости создания систем компьютерного зрения, которые имеют лучшую эффективность, высокую скорость обработки данных для смартфонов. Система должна соответствовать условиям работы в среде шумов и в реальном времени. Развитие этого направления может быть использовано для создания интеллектуальных приложений в областях моды, красоты, фитнеса, анализа биомедицинских изображений.

Целью диссертации является разработка алгоритмов, методов компьютерного зрения в извлечении признаков и классификации антропометрических данных для построения зрительной системы для смартфонов. Эта система используется в условиях шумов и обработки данных статических изображений и видеокадров в реальном времени.

Для достижения этих целей необходимо решить следующие задачи:

\begin{itemize}
	\item Выполнять анализ подходов с использованием алгоритмов, методов компьютерного зрения для извлечения антропометрических признаков с изображений и видеопоследовательностей и предложить наиболее подходящий подход для решения данной задачи;
	\item Исследование и применение методов калибровки для повышения точности сегментации и местонахождения, измерять размеры частей человеческого тела на изображениях и видеопоследовательностях;
	\item Разработка и реализация алгоритмов, программы извлечения и классификации антропометрических данных на основе алгоритмов, методов предложенного компьютерного зрения;
	\item Оценка и редактирование точности результата системы компьютерного зрения в антропометрии;
	\item Разработка приложений компьютерного зрения в антропологии для пошива промышленности и здоровья, фитнеса. Система работает со статическими изображениями и видео в режиме реального времени;
	\item Сравнение результатов зрительной системы в антропометрии с полученными результатами других авторов.
\end{itemize}