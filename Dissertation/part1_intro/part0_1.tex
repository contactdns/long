%\section{Пример из моделирования развивающихся динамических систем} \label{sect3_2}
\section{Задачи и развитие систем компьютерного зрения} \label{chapter1.1}

Машинное зрение – это междисциплинария область получившая в настоящее время широкое развитие. Концепция обработки изображений и машинного зрения является важным разделом компьютерного зрения и основана на комбинации многих дисциплин. Прогресс в области компьютерного зрения определяется развитием математической теории, численных методов и развитием аппаратного обеспечения. Долгое время теоретические исследования опережали возможности практического использования систем компьютерного зрения. Условно здесь можно выделить следующие этапы  развития \cite{Paragios2008}. Начиная с 1970-х годов, когда вычислительная мощность ВТ становится все более сильной, появляется возможность обрабатывать большие наборы данных, таких как изображения, звук и видео. Концепции и методы машинного зрения привлекают все большее внимание.

Системы машинного зрения включают в себя следующие основные компоненты \cite{Rosenfeld2000}: 

\begin{itemize}
	\item Подсистему формирования изображений, которая сама может включать разные компоненты, включая оптическую систему, осветщение и ПЗС- или КМОП-матрицу;
	\item Вычислитель;
	\item Алгоритмы анализа изображений, которые могут реализовываться программно на процессорах общего назначения, аппаратно в структуре вычислителя и даже аппаратно в рамках подсистемы формирования изображений.
\end{itemize}

Машинное зрение предоставляет важнейшую информацию для создания систем искусственного интеллекта. Такие системы могут получать информацию как из полученных  изображений так и из наборов многомерных данных различной природы \cite{Fan2013}. Сочетание машинного зрения с другими областями, такими как: информационные технологии, связь, электроника, автоматическое управление и т.д. дает нам множество применений в области науки, безопасности, военной , медицины, промышленных роботов и т.д.


В последние годы проводятся научно-исследовательские работы в различных областях обработки и распознавания изображений. Машинное зрение стало самостоятельной дисциплиной \cite{Huang1991}, \cite{Kevin1991}. В настоящее время методы машинного зрения реализованы во множестве устройств, применяются технологии обработки и управления, на основе изображения. Приведем краткий обзор приложений систем машинного зрения в робототехнике.

\subsection{Компьютерное зрение в робототехнике}

Системы компьютерного зрения роботов является интегрированной системой может быть одной или нескольких камер \cite{Abebe2016}. В зависимости от назначения, система должна обнаруживать положение объектов в поле зрения камеры (ROI, Region of Interest). На основе этой системы, робот может определить местоположение и направление движения объекта. Для достижения высокой эффективности работы необходимо обеспечивать эффективную совместную работу системы, включая аппаратные средства и программное обеспечение \cite{Zhu2004}, \cite{Vayda1991}.

В современной промышленности системы компьютерного зрения роботов применяется в различных областях, таких как:

\begin{itemize}
	\item автомобильная промышленность с автоматизированными системами сборки и обработки двигателей, автомобильных кузовов;
	\item пищевая промышленность. Например, система контроля автоматического закрывания пакета, закрывания контейнера  обнаружения посторонних предметов в пище при упаковке;
	\item фармацевтическая промышленность: контроль упаковки и партии, обнаружение дефектов;
	\item военн силовые структуры: обнаружение целей беспилотниками, анализ багажа, анализ поведения групп людей в общественных местах имеет система дистанционного управления беспилотников.
	\item обеспечение безопасности, предупреждение преступности в том числе на основе антропометрических признаков;
	\item индустрии развлечений  и спорта: системы автоматического управления камерой слежения за объектами в футбольном матче, гонках и т.д.
\end{itemize}

Таким образом, современная робототехника требует решения многих сложных задач компьютерного зрения, которые ставит перед собой в том числе следующие задачи:

\begin{itemize}
	\item Задачи, связанные с ориентацией во внешнем пространстве, определением расстояний до объектов и т. д;
	\item Задачи по распознаванию  и классификации различных объектов и интерпретации сцен в целом задачи навигации;
	\item Задачи по обнаружению людей, распознаванию их лиц и анализу эмоций;
	\item Задачи восстановления изображений и подавления шумов. Повышение разрешнения изображений (сверхразрешение).
\end{itemize}

\subsection{Машинное обучение и информационный поиск}

На сегодняшний день машинное обучение используется в бесчисленном множестве интернет-проектов. У многих известных IT-компаний (к примеру, Яндекс, Google, Facebook, Microsoft) на нём базируются многие ключевые технологии \cite{Cormier2016}.

Задачи поиска изображений по содержанию также разнообразны. Они включают в себя следующие шаги \cite{Meer2000}. Сравнение содержимого изображения для обнаружения и распознавания на изображениях объектов классов разной степени общности. Такие алгоритмы очень полезны для создания приложений, таких как: классификация данных (изображения, видео и т.д.), поиск товаров на основе изображений для интернет-магазинов, для извлечения изображений в геоинформационных системах, для систем биометрической идентификации, для специализированного поиска изображений в социальных сетях (например, для поиска лиц людей, привлекательных для пользователя) \cite{Findface} и т.д., вплоть до поиска изображений в интернете.

Методами машинного обучения при разработке систем компьютерного зрения решаются проблемы распознавания объектов. Разработка алгоритмов классификации является одной из важнейших областей машинного обучения \cite{Murino2000}. Методы глубокого обучения (deep learning) требуют огромных вычислительных ресурсов, и даже для обучения распознаванию ограниченного класса объектов могут требоваться несколько дней работы на вычислительном кластере. При этом в будущем могут быть разработаны еще более мощные, но требующие еще больших вычислительных ресурсов методы.  Отметим, что использование специфики решаемой задачи позволяет существенно сократить вычислительную сложность, однако требует более глубокого понимания сути решаемой задачи.  

\subsection{Мобильные приложения}

Задачи компьютерного зрения все шире используются в приложениях для персональных мобильных устройств, таких как смартфоны, планшеты и прочее. В частности, число смартфонов неуклонно растет и уже практически превысило по численности население земли \cite{Battiato2012}. Часть задач по обработке изображений для мобильных устройств с камерами совпадает с задачами для цифровых фотоаппаратов. Основное отличие заключается в качестве объективов и в условиях съемки. В спектре аппаратного обеспечения, доступного для решения задач отметим модули с доступом к интернет, наличия интернета, GPS и конечно мощного процессора и большой памяти. Именно с этим связано появление такого термина, как <<smart phone>> \cite{Hannuksela2007}. 

В этой связи решение, казалось бы, идентичных задач для разных устройств может различаться, что делает эти решения высоко востребованными на рынке. Приложения для смартфонов выходят на рынок, наблюдается огромный интерес к компьютерному зрению и приложений дополненной реальности в мобильных устройствах \cite{Shubina2010}. 


В настоящее время существует много приложений обработки изображений для мобильных устройств. Например: программа автоматической коррекции лицевых дефектов (таких, как морщины, угри, веснушки) в соответствии со стилем: натуральный, классический, контраст и т.д. Это такие приложени, как camera360 , Perfect Selfie и прочее. Приложения по обработке изображений на видео позволяют пользователям создавать тематические видео из фотографий, анимации и т.д.  Как: MiniMovie, Imovie-editing и Маскрарад - преобразователь видео в стиле картин художника Ван Гога и мультипликации. Таким образом, класс задач компьютерного зрения, решения которых могут быть применены в мобильных приложениях, крайне широк. Большой набор приложений есть у методов сопоставления (отождествления сопряженных точек) изображений, в том числе с оценкой трехмерной структуры сцены и определением изменения ориентации камеры и методов распознавания объектов, а также анализа лиц людей. Имеется широкий спектр мобильных приложений для которых будет требоваться разработка специализированных методов компьютерного зрения.