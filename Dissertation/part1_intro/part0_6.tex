\section{Основные результаты и выводы по главе 1}

\begin{enumerate}
	\item В этой главе анализируются алгоритмический подход, методы компьютерного зрения в извлечении антропометрических признаков со статических изображений и видеопоследовательностей:
	
	\begin{itemize}
		\item Построение признаков ключевых точек из 2D-изображений, видеокадров;
		\item Использование 3D-сканирования моделей и сопоставление точек человеческого тела в построенных моделях.
	\end{itemize}
	
	\item В данной главе анализируется алгоритмический подход, методы компьютерного зрения в классификации антропометрических данных признаков в изображениях и видеопоследовательностях. Включая:
	
	\begin{itemize}
		\item Алгоритм Adaboost;
		\item Нейронные сети;
		\item Опорных векторов (SVM).
	\end{itemize}
\item Также здесь рассматриваются возможности использования алгоритмов ICP и разреза графов (Graph-сuts) для извлечения антропометрических признаков и алгоритм случайного леса для классификации антропометрических данных в изображениях и видеопоследовательностях.
\item На основе проведенного анализа принято решение о адекватности и точности использования комбинированных алгоритмов ICP, разреза графов (Graph-cuts) для извлечения антропометрических признаков с использованием метода случайного леса (Random Forest) для классификации антропометрических данных в статических изображениях, видео в присутствии шума и в режиме реального времени. Такой подход позволяет построить систему компьютерного зрения в антропометрии с высокой точностью и скоростью обработки.
\end{enumerate}
