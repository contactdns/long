\section{Приложение компьютерного зрения в антропометрии}

Применение методов компьютерного зрения в антропометрии должны соответствовать следующим требованиям: быстрая скорость обработки, высокая точность, возможность создать базу данных для создания приложений на практике. Для приложений безопасности, например, используется программное обеспечение распознавания объектов на основе антропометрических признаков лиц \cite{Graham1997}, 3D моделирование \cite{Zouhour2006} для областях тестильной промышленности, здоровья. Сканеры тела все чаще используются для оценки состояния здоровья и антропометрии, напримеры Caesar project \cite{Robinette2006}, SizeUK \cite{Uk2013} and SizeUSA \cite{Usa2013}. В будущем такие системы будет развиваться для решения различных задач антропометрии.

BreuckmannBodySCAN \cite{Hexagon2016} - это система для проведения антропометрических измерений. По сравнению с ручным измерением, антропометрическое измерение демонстрирует незначительную погрешность. Однако это показывает, что автоматические антропометрические измерения, сравнимо лучше традиционного метода - вручную. Для проверки ошибок авторы используют систему на основе так называемого соотношения значительного/границы (significant/borderline). Если коэффициент > 0.9, то результат измерения является приемлемым. Преимущество этой системы состоит в том, что она не зависит от движения объекта или отсутствия движения объекта, не подверженных внешним факторам, таких как шум и т.д.

3DscannerData \cite{Sobota2009} является приложением, в котором система компьютерного зрения используется в области медицины. Система анализирует форму человеческого тела. Система позволяет проводить ряд антропометрических измерений на основе данных со сканера человеческого тела. Система позволяет анализировать и прогнозировать жир в организме человека. Системы, основанные на размере тела, по сравнению со стандартным размером, чтобы выяснить разницу. Недостатком этой системы является то, что нельзя определить точное местоположение ключевых точке на теле человека для проведения антропометрических измерений. Для повышения точности системы, авторы улучшили алгоритм обнаружения позиций учета каждой части тела. Преимущества вычислительной системы заключается в быстрой работе и не нуждается во вмешательстве  пользователя.

Некоторые интеллектуальные системы на основе компьютерного зрения были применены в текстильной промышленности. Среди них, уже существующих систем были использованы непосредственно для изготовления одежды на заказ. Поэтому, в рамках диссертации предлагается система, которая, например, может помочь портному (или сотруднику фитнес-центра) измерить размеры тела клиента автоматически. Эта система использует способ и систему, основанную на  компьютерном зрении, обеспечивая комфорт и конфиденциальность для клиентов и в целях экономии средств и удобств для заказчика такого антропометрического измерения. С помощью экспериментов, система доказала, что имеет возможность измерить размеры частей человеческого тела быстро и качественно, даже при наличии шума в исходных данных. Таким образом, система автоматического измерения человеческого тела на основе компьютерного зрения представяляет интерес в различных областях, включая швейную промышленность.

Кроме того, эта система может применяться в области фитнеса. Система помогает пользователям часто проводить антропометрическое измерение, проверкять индекс ожирения, выдавая в результате. Таким образом, работы состоит в том, чтобы на основе методов компьютерного зрения создать приложение для смартфонов для решения задач антропометрии.