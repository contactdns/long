%%% Колонтитулы %%%
\usepackage{fancyhdr}

%%% Прикладные пакеты %%%
\usepackage{calc}               % Пакет для расчётов параметров, например длины
%\usepackage{etoolbox}          % ради функции patchcmd для управления списком литературы

\usepackage {interfaces-base}   % Набор базовых интерфейсов к некоторым пакетам, конкретные реализации загружаются в стиле

%%% Заголовки %%%
\usepackage{titlesec}           % Пакет настройки шрифтов заголовков в тексте

%%% Оглавление %%%
\usepackage{tocloft}

%%% Счётчики %%%
\usepackage{chngcntr}           % оперативная перенастройка счётчиков


\usepackage{xcolor}
\usepackage[version=1,draft]{pdfcomment}% FIXME: Remove draft before making the final layout!!!
\definecolor{myblue}{rgb}{0.045,0.278,0.643}
\colorlet{myorange}{red!30!yellow}
\defineavatar{EA}{color=myorange,author={Evgeny Cherkashin},opacity=0.7}%
\defineavatar{DN}{color=myblue,author={Dennis Sidorov},opacity=0.7}%
\newcommand{\cEA}[1]{\pdfcomment[avatar=EA]{#1}}
\newcommand{\cDN}[1]{\pdfcomment[avatar=DN]{#1}}

\usepackage[draft,obeyDraft]{todonotes}% Another comment engine
\newcommand{\mrk}[1]{\todo[inline,color=green,size=\footnotesize]{#1}{}}
% \todo[inline]{This is a todo note inline}
% \todo{This is a to do note at margin}
% listoftodos

% These are examples of annotation icons
%\pdfcomment[subject={Top2},author={Daisy Duck},color={0.234 0.867 0.211},voffset=8pt,opacity=0.5]{This is a comment.}  %\pdfcomment[id=1,color=myblue,subject={Comment2},icon=Note,open=true,hspace=100pt]{This is another comment.}

%These are'nt compatible to the dissertation text
%\usepackage[sort&compress]{natbib} % Adjusting \cite's
%\setcitestyle{numbers,square,comma}

%%% Local Variables:
%%% mode: latex
%%% TeX-master: "../dissertation"
%%% End:
