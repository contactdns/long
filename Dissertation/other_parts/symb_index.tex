\chapter*{Основные обозначения}
\addcontentsline{toc}{chapter}{Основные обозначения}
%\noindent {\bf Основные обозначения}\\


\noindent $\mathbb{R}^n$ -- множество  действительных $n$-мерных векторов\\

\noindent $\mathbb{N}$ -- множество  натуральных чисел\\

\noindent $E_1,\, E_2$ -- банаховы пространства\\

\noindent  $\mathcal{L}(E_1 \rightarrow  E_2) $ -- множество линейных ограниченных операторов, действующих из $E_1 $ в $E_2$\\

\noindent  $A\in \mathcal{L}(E_1 \rightarrow  E_2) $ --  линейный ограниченный оператор  из $E_1 $ в $E_2$\\


\noindent $L_p(a,b)$ -- пространство Лебега с нормой $||x||_{L_p} = \sqrt[p]{\int_a^b |x(t)|^p \, dt}$ \\

\noindent $D(B)$  -- область определения оператора $B$\\

\noindent $R(B)$ -- область значений оператора $B$\\


\noindent\hspace*{-0.4cm}$\stackrel{\hspace{3.9mm} \circ\,(n)}{\mathcal{C}}_{\hspace{-5mm}\,\,[0,\sc T]} $ -- пространство $n$ раз непрерывно дифференцируемых функций $x(t)$, заданных на компакте $[0,T],\, x(0)=0$\\

\noindent $\mathcal{C}_{[0,T]}$ --  пространство  непрерывных функций, заданных на компакте $[0,T]$ \\

\noindent $\mathcal{L}(\mathcal{C}_{[0,T]} \rightarrow \mathcal{C}_{[0,T]}) $ -- пространство линейных непрерывных операторов, действующих из $\mathcal{C}_{[0,T]} $ в $ \mathcal{C}_{[0,T]}$\\

\noindent $\mathcal{C}_{([0,T];E)}, $  $\mathcal{C}_{[0,T]}^{E}$ --  пространство абстрактных непрерывных функций, заданных на компакте [$0,T]$ со значениями в  банаховом пространстве  $E$\\

\noindent $\mathcal{C}_{[0,T]}^{+} $ --  пространство  положительных непрерывных функций, заданных на компакте  $[0,T]$\\

\noindent ${\overline{D}}$ -- замыкание области $D$\\

%\noindent $N(B)$  -- множество решений однородного операторного уравнения $Bx=0$\\

\noindent $\text{dim} N(B)$ -- число линейно независимых решений однородного уравнения  $Bx=0$\\ 

\noindent $B^*$ -- сопряженный оператор\\

\noindent $B^* \in \mathcal{L}(E_2^* \rightarrow  E_1^*) $ --  сопряженный оператор
действующий  из $E_2^* $ в $E_1^*,$ где $E_2^*,$ $E_1^*$ -- пространства, сопряженые к $E_2$ и $E_1.$\\

\noindent $N(B), \, Ker B$  -- множество решений  линейного однородного (операторного)  уравнения $B x=0$\\

\noindent  $S(0,r)$ -- шар в нормированном пространстве с центром в нуле и радиусом $r$\\

\noindent $\langle \cdot , \gamma\rangle $ -- действие функционала $\gamma$
как элемента из сопряженного пространства $E^*$ на элемент из пространства $E$\\


\noindent $\langle x , \gamma\rangle $ -- результат этой операции над элементом $x$ из $E$\\


\noindent$||\cdot||_{\mathcal{L}(\mathbb{R}^n \rightarrow \mathbb{R}^n )}, \,  |\cdot|_{\mathcal{L}(\mathbb{R}^n \rightarrow \mathbb{R}^n )}$ -- норма матрицы размерности $n \times n$\\

\noindent$||\cdot||_{\mathbb{R}^n}, \,  |\cdot|_{\mathbb{R}}$ -- норма вектора размерности $n$\\

\noindent$|| \cdot ||_{\mathcal{L}(E_1\rightarrow E_2)} $ -- норма линейного оператора, действующего из пространства $E_1$ в $E_2$ \\

\noindent$|| \cdot ||_{\mathcal{L}({\mathcal{C}_{[0,h]}}\rightarrow {\mathcal{C}_{[0,h]}})} $ -- норма линейного оператора, действующего из пространства ${\mathcal{C}_{[0,h]}}$ в ${\mathcal{C}_{[0,h]}}$ \\

\noindent $C_n^m, \, \left(\begin{array}{c} n\\m\end{array}
	\right)$ -- число сочетаний из $n$   по $m,$ $({n!}/{(n-m)! m!})$\\

\noindent $e(t)$ -- функция  Хевисайда\\

\noindent $\delta(t)$ -- функция Дирака\\

\noindent $\text{supp}$  -- носитель функции\\

\noindent ${\mathcal{D}}_{(0,T)}$ -- множество бесконечно дифференцируемых финитных функций с носителями  на интервале $(0,T)$\\

\noindent ${\mathcal{D}}_{\left(0,T\right)}^{'} $ -- множество линейных непрерывных функционалов (обобщенных функций) $\gamma,$ определенных на $D,$ $\gamma \in \mathcal{L}({\mathcal{D}}_{\left(0,T\right)} \mathop{\to }\limits^{} (-\infty ,+\infty ))$ \\

\noindent ${\mathcal{D}}_{n\, (-\rho,\rho)}^{\prime}$ -- линейное множество  элементов вида $c_0\delta(t) + c_1\delta^{(1)}(t) + \cdots$\\ $\cdots + c_n\delta^{(n)}(t) + u(t), $
где $u(t) \in \mathcal{C}_{(-\rho, \rho)}$

%\noindent $(D;E^*)$ -- множество бесконечно-дифференцируемых финитных функций  с носителями в интервале
%$(0,\rho)$ и значениями в $E^*$\\
%
%\noindent $(D^{\prime};E)$ -- множество линейных непрерывных функционалов, определенных на $(D;E^*)$ со значениями в $E$ (пространство обобщенных функций)\\


