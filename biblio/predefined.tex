%%% Реализация библиографии встроенными средствами посредством движка bibtex8 %%%

%%% Пакеты %%%
\usepackage{cite}                                   % Красивые ссылки на литературу


%bibtex8.exe "%tm" --huge --csfile "cp1251rus.csf
%bibtex8.exe "%tm" --huge --csfile "utf8cyrillic.csf

%%% Стили %%%
%\bibliographystyle{../BibTeX-Styles/utf8gost71u}    % Оформляем библиографию по ГОСТ 7.1 (ГОСТ Р 7.0.11-2011, 5.6.7)
%\bibliographystyle{../BibTeX-Styles/utf8gost71s}    % Оформляем библиографию по ГОСТ 7.1 (ГОСТ Р 7.0.11-2011, 5.6.7)
%\bibliographystyle{../BibTeX-Styles/gost2003smed}    % Оформляем библиографию по ГОСТ 7.1 (ГОСТ Р 7.0.11-2011, 5.6.7)
%\bibliographystyle{../BibTeX-Styles/ugost2003s}    % Оформляем библиографию по ГОСТ 7.1 (ГОСТ Р 7.0.11-2011, 5.6.7)
\bibliographystyle{../BibTeX-Styles/ugost2003}    % Оформляем библиографию по ГОСТ 7.1 (ГОСТ Р 7.0.11-2011, 5.6.7)
%\bibliographystyle{../BibTeX-Styles/ugost2003s_}    % Оформляем библиографию по ГОСТ 7.1 (ГОСТ Р 7.0.11-2011, 5.6.7)
%\bibliographystyle{../BibTeX-Styles/gost2003med_utf}    % Оформляем библиографию по ГОСТ 7.1 (ГОСТ Р 7.0.11-2011, 5.6.7)
%\bibliographystyle{../BibTeX-Styles/ugost2008s}    % Оформляем библиографию по ГОСТ 7.1 (ГОСТ Р 7.0.11-2011, 5.6.7)

\makeatletter
%\renewcommand{\@biblabel}[1]{#1.}   % Заменяем библиографию с квадратных скобок на точку
\makeatother
%% Управление отступами между записями
%% требует etoolbox 
%% http://tex.stackexchange.com/a/105642
%\patchcmd\thebibliography
% {\labelsep}
% {\labelsep\itemsep=5pt\parsep=0pt\relax}
% {}
% {\typeout{Couldn't patch the command}}

%%% Цитирование %%%
\renewcommand\citepunct{;\penalty\citepunctpenalty%
    \hskip.13emplus.1emminus.1em\relax}                % Разделение ; при перечислении ссылок (ГОСТ Р 7.0.5-2008)
%

%%% Создание команд для вывода списка литературы %%%
\newcommand{\biblioildar}{../biblio/bases/biblio_long}
%\newcommand{\biblioildar}{../biblio/bases/biblio_ildar_only}
%\newcommand{\bibliosidorovdn}{../biblio/bases/biblio_sidorovdn}
%\newcommand{\bibliosidorovdn}{../biblio/bases/for_synopsis/biblio_ildar_vak}
%\newcommand{\biblioildar}{../biblio/bases/for_synopsis/biblio_ildar_non_vak}
%\newcommand{\biblioildar}{../biblio/bases/biblio_ildar_only_cp1251}
%\newcommand{\bibliosidorovdn}{../biblio/bases/biblio_sidorovdn_cp1251}

\newcommand*{\insertbibliofull}{
%\bibliography{../biblio/bases/othercites,../biblio/bases/authorpapersVAK,../biblio/bases/authorpapers,../biblio/bases/authorconferences}         % Подключаем BibTeX-базы % После запятых не должно быть лишних пробелов — он "думает", что это тоже имя пути
%\bibliography{\biblioildar,\bibliosidorovdn}
%\bibliography{../biblio/bases/biblio_ildar_only,../biblio/bases/biblio_sidorovdn}
\bibliography{../biblio/bases/biblio_long}
%\bibliography{../biblio/bases/for_synopsis/biblio_ildar_vak,../biblio/bases/for_synopsis/biblio_ildar_non_vak}
}

%\newcommand*{\insertbiblioauthor}{
%%\bibliography{../biblio/bases/authorpapersVAK,../biblio/bases/authorpapers,../biblio/bases/authorconferences}         % Подключаем BibTeX-базы % После запятых не должно быть лишних пробелов — он "думает", что это тоже имя пути
%\bibliography{\biblioildar}
%%\bibliography{../biblio/bases/biblio_ildar_only}
%}
%
%\newcommand*{\insertbiblioother}{
%%\bibliography{../biblio/bases/othercites}         % Подключаем BibTeX-базы
%\bibliography{\bibliosidorovdn}
%%\bibliography{../biblio/bases/biblio_sidorovdn}
%}

\newcommand*{\insertbibliovak}{
%\bibliography{../biblio/bases/othercites,../biblio/bases/authorpapersVAK,../biblio/bases/authorpapers,../biblio/bases/authorconferences}         % Подключаем BibTeX-базы % После запятых не должно быть лишних пробелов — он "думает", что это тоже имя пути
%\bibliography{\biblioildar,\bibliosidorovdn}
\bibliography{../biblio/bases/for_synopsis/biblio_ildar_vak}
}

\newcommand*{\insertbibliononvak}{
%\bibliography{../biblio/bases/othercites,../biblio/bases/authorpapersVAK,../biblio/bases/authorpapers,../biblio/bases/authorconferences}         % Подключаем BibTeX-базы % После запятых не должно быть лишних пробелов — он "думает", что это тоже имя пути
%\bibliography{\biblioildar,\bibliosidorovdn}
\bibliography{../biblio/bases/for_synopsis/biblio_ildar_non_vak}
}

%% Счётчик использованных ссылок на литературу, обрабатывающий с учётом неоднократных ссылок
%% Требуется дважды компилировать, поскольку ему нужно считать актуальный внешний файл со списком литературы
\newtotcounter{citenum}
\def\oldcite{}
\let\oldcite=\bibcite
\def\bibcite{\stepcounter{citenum}\oldcite}
